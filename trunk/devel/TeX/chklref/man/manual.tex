\documentclass[a4paper,11pt,twoside]{article}
\usepackage{a4wide}
\usepackage{latexsym}
\usepackage{amssymb, amsmath, amsthm}
\usepackage{fancyhdr}
\usepackage[latin1]{inputenc}
\usepackage{enumitem}
\usepackage[round]{natbib}
\usepackage{times}
\def\cite{\citet}

%%--------------------------------------------
\hbadness=10000
\emergencystretch=\hsize
\tolerance=9999
\textheight=9.0in
%%--------------------------------------------

\usepackage[english]{babel}
\def\chk{{\it chklref}}



\title{chklref\\
{\large version 1.0}}
\date{\today}
\author{J\'er\^ome Lelong\footnote{Email:\texttt{jerome.lelong@gmail.com}}}
  

\begin{document}
\maketitle

\begin{abstract}
  Which \LaTeX\ user has never dreamt of cleaning its \TeX\ file of any
  unneeded labels? This is precisely what \chk\ aims to do. 
\end{abstract}

\section{What {\it chklref} can do for a \LaTeX\ user ?}

While typing a \TeX, it is quite common to add labels that will in the end
reveal unneeded, this is especially true for equations or any other mathematical
environments. It would be quite convenient to wipe out those labels from the
final version of the document. \chk\ has been designed to help any \TeX\
user doing so.

\chk\ parses all the \TeX file to detect labels that do not have a matching
reference. At the end of the program, a summary is printed with the list of
labels to be removed. Obviously, this search is performed in the whole document
and not only within mathematical environments. Nonetheless, there is something
specifically dedicated to mathematical environments : it looks for the non
starred mathematical environments not containing any label.



\section{Usage}

Typical usage : \verb!chklref file.tex!.

\section{Requirements}

\chk\ is written in Perl and should work with any standard Perl installation.

\end{document}
