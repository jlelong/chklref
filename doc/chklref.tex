\documentclass[a4paper,11pt,twoside]{article}
\usepackage{a4wide}
\usepackage{latexsym,t1enc,lmodern,url}
\usepackage{amssymb, amsmath}
\usepackage[latin1]{inputenc}
\usepackage[english]{babel}

%%--------------------------------------------
\hbadness=10000
\emergencystretch=\hsize
\tolerance=9999
\textheight=9.0in
%%--------------------------------------------

\def\chk{{\it chklref}}

\makeatletter

\title{Check \LaTeX\ references : chklref \\
Version \chkVERSION}
\date{\today}
\author{J\'er\^ome Lelong}
  

\begin{document}
\maketitle

\begin{abstract} Which \LaTeX\ user has never dreamt of cleaning its \TeX\ file
  of any useless labels right before submitting an article for instance? This is
  precisely what \chk\ will help you achieve.
\end{abstract}

\section{What {\it chklref} can do for a \LaTeX\ user ?}

While writing a \LaTeX document, it is quite common to add labels that will 
reveal unused in the end, this is especially true for equations or any other
mathematical environments. It would be quite convenient to wipe out those labels
from the final version of the document. \chk\ has been designed to help any
\TeX\ user do so.

\chk\ parses the \LaTeX\ file to detect labels that do not have any matching
reference. At the end of the program, a summary is printed with the list of
labels to be removed. Obviously, this search is performed in the whole document
and not only within mathematical environments. Nonetheless, there is something
specifically dedicated to environments that have a star version: it looks
for the non star environments not containing any label.

For instance, consider you a have a \LaTeX\ file with the following piece of code
\begin{verbatim}
\begin{equation}
  x^2 + x + 1 = 0
\end{equation}
\end{verbatim}
\chk\ will advise you to consider using \verb!equation*! instead of
\verb!equation! in order to remove the unneeded number appearing next to the
equation.

Star environments with labels are also tracked because with most \LaTeX\
distributions it does not even raise a warning. However in case you may refer to
this label the number that would appear is likely to be nonsense.

The summary printed at the end of the program contains two sections (see
Figure~\ref{fig:output}): one is about labels in star environments and one is about
unneeded labels. Each of these output lines begins with a line number referring
to the parsed \LaTeX\ file. Then, the decision to be made is printed:
``remove label'' or ``use a starred environment''. In the case of a ``remove
label'' message, the label to be removed is also printed to avoid any possible
confusion.

\begin{figure}[htbp]
  \centering
\begin{verbatim}
************************************
** Labels in starred environments **
************************************
line  264 : remove label eq:quantified
line  200 : consider using a STAR environment

*******************
** Unused Labels **
*******************
line  313 : remove label eq:prog_dyn_prix
\end{verbatim}
  \caption{Program output}
  \label{fig:output}
\end{figure}

Since version 2.5, the checking of bibliography entries has been added to
detect if some entries are not explicitly cited in the text. In this case, a
``remove bibitem'' message is printed.
\begin{verbatim}
**********************************
** Non cited Bibliography entries **
**********************************
remove bibitem : premia
\end{verbatim}


\section{Usage}

Typical usage : \verb!chklref file.tex!.
\\


\section{Requirements}

\chk\ is written mainly in \TeX with a small Perl script to parser the output
generated by the \TeX\ package. It should be working with any standard Perl
installation.

\section{Source code}

The source code is available from GitHub
\url{https://github.com/jlelong/chklref}.
\\

\noindent It is distributed under the GPL licence version 3 or any later version
in the hope it will be useful but without any warranty. Should you find a bug in
chklref, send me a mail to \verb!jerome.lelong@gmail.com! describing the problem
with a minimal \LaTeX\ file to reproduce the problem or fill in an issue on the
GitHub repository.

\end{document}
% vim: set spelllang=en spell:

